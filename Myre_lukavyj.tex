\documentclass{song}
\usepackage[czech]{babel}

\title{Myre lukavyj}
\author{ukrajinská}

\begin{document}

\strophe
Em      D       G             D   Em
|Myre lu|kavyj, |skorbmy ispol|něn|nyj!
Em        D         G           D   Em
|Koľ ty ne|tverdyj, |koľ nesover|šen|nyj!
G                 D
\[ |Koľ suť ne blahi |tvoji zdi utichy,
   Em                Hm
 |koľ suť plačevni |radosti i smichy,
G     D     Hm  Em
|rados|ti i |smi|chy. \]
\endstrophe

\strophe*
Koľ nespokojni tvoji česť, bohatstvo,
vitr, dym, ničtože, vse nepostojanstvo.
\[ Cvite tut v jedyn čas, v druhyj uvadaje,
dnes na prestoli -- \[ zavtra nyzpadaje. \] \]
\endstrophe

\strophe*
Chde jesť horďajsja Davyd vopijaše
vse mymo ide, uže toj ne bjaše.
\[ Mnyťsja nam sladki tvoji zdi zabavy
slasťmy iznurenni \[ suť našiji nravy. \] \]
\endstrophe

\strophe*
Pochoti smyslu zrity prepjat stvujuť,
horkoje sladko byty pokazujuť.
\[ Vysokij suščij vsjak v myri hordycja,
mňat že bohati \[ što smerť jich bojicja. \] \]
\endstrophe

\strophe*
Što nam v bohatsvi ašče smerť carstvuje,
in sobyraje, in naslidstvuje.
\[ Što nam za poľza ot česti byvaje
jehda smerť ranhy \[ vesma prezyraje. \] \]
\endstrophe

\strophe*
Jedyna ubo mysľ virnym da bude,
kak Sudija v toj den strašnyj prybude!
\[ Sprosjat nas bidnych tam vskori otvita --
počto terjaly \[ vsuje naši lita! \] \]
\endstrophe

\strophe*
Myre lukavyj, skorbmy ispolněnnyj!
Koľ ty netverdyj, koľ nesoveršennyj!
Koľ suť ne blahi tvoji zdi utichy,
\[ koľ suť plačevni \[ radosti i smichy. \] \]
\endstrophe

\end{document}
