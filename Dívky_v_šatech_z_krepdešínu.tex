\documentclass{song}
\usepackage[czech]{babel}
\usepackage{multicol}

\author{Jarek Nohavica}
\title{Dívky v~šatech z~krepdešínu}

\begin{document}

\begin{multicols}{2}
\strophe
Am           E7
|Dívky chodí |v~šatech z~krepdešínu
Am                    E7
|a já se z~těch dívek |div nepominu
Dm
|oči navrch hlavy
E7                 Am,E7
|já nechci do Opavy|
\endstrophe

\columnbreak

\strophe*
Co má být ukryto ať je kruci skryto
kdopak má vydržet tohleto neurčito
tyhlety siluety
devatenáctiletý
\endstrophe
\end{multicols}

\refrain
C       G
|Sunday |Monday
Am                        E7
|klapky na oči si chlapče |nandej
                F
zatracený krepde|šín
E7               Am
|život už mě netě|ší
\endstrophe

\begin{multicols}{2}
\strophe*
Představa co by bylo kdyby bylo
no schválně koho by to nezlomilo
a~tak teď sebou hážu
a~usnout nedokážu
\endstrophe

\strophe*
Jen jsem prošel městem a jsem groggy
zavolejte prosím psychology
trpí moje ego
krucinálhimlhergot
\endstrophe
\end{multicols}

\ref{} \ldots{} zatracený psycholog, copak já jsem ňákej cvok

\begin{multicols}{2}
\strophe*
Na vině je moje fantazie
nedá mi pokoje furt do mě ryje
kreslí černou tuší
co jenom matně tuším
\endstrophe

\strophe*
A~já abych řek pravdu toho tuším hodně
fantazie kreslí věrohodně
troufne si na cokoli
má totiž na to školy
\endstrophe
\end{multicols}

\ref{} \ldots{} zatracená fantazie, člověk klidně nepožije.

\strophe*
Slunce koncem léta bývá rozžhavené
a holky v~krepdešínu jsou jako pod rentgenem
lepší je doma seděti
máš-li ženu a dvě děti
\endstrophe

\end{document}
