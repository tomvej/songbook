\documentclass{song}
\usepackage[czech]{babel}
\usepackage{multicol}

\title{Trubadůrská}
\author{Karel Plíhal}

\begin{document}

\begin{multicols}{2}

\strophe
Am        Em        Am
|Od hradu |ke hradu |putujem,
C        G        E\7
|zpíváme |a holky |muchlujem.
   Am      Am/G Fmaj\7\,Fmaj\7/F\#
\[ |Dřív ji|nam |nejedem,
Am        Em        Am
|dokud tu |poslední |nesvedem. \]
\endstrophe

\strophe*
Kytary nikdy nám neladí,
naše písně spíš kopnou než pohladí,
\[ nakopnou zadnice
ctihodných měšťanů z radnice. \]
\endstrophe

\refrain
G            C\,Em\7\,Am\,Am/G
|Hop hej, je |veselo,
    Fmaj\7 Fmaj\7/F\# G
pan |kníže |pozval |kejklíře,
G            C\,Em\7\,Am\,Am/G
|hop hej, je |veselo,
     Fmaj\7  Em    Am
dnes |vítaní |jsme |hosti.
G            C\,Em\7\,Am\,Am/G
|Hop hej, je |veselo,
   Fmaj\7  Fmaj\7/F\# G
ač |nedali |nám |talíře,
G            C\,Em\7\,Am\,Am/G
|hop hej, je |veselo,
    Fmaj\7    Em     Am
pod |stůl nám |hážou |kosti.
\endstrophe

\strophe*
Nemáme způsoby knížecí,
nikdy jsme nejedli telecí,
\[ spáváme na seně,
proto vidíme život tak zkresleně. \]
\endstrophe

\strophe*
A doufáme, že lidi pochopí,
že pletou si na sebe konopí,
\[ že hnijou zaživa,
když brečí v hospodě u piva. \]
\endstrophe

\ref{}

\columnbreak

\strophe*
Ale jako bys lil vodu přes cedník,
je z tebe nakonec mučedník,
\[ čekaj' tě ovace
a potom veřejná kremace. \]
\endstrophe

\strophe*
Rozdělaj' pod náma ohýnky
a jsou z toho lidové dožínky.
Kdo to je tam u kůlu,
ale příliš si otvíral papulu.
Kdo to je tam u kůlu,
borec, za nás si otvíral papulu.
\endstrophe

\ref{}

\strophe*
Od hradu ke hradu putujem,
zpíváme a holky muchlujem.
\[ Dřív jinam   nejedem,
dokud tu poslední nesvedem. \]
\endstrophe

\strophe*
To radši zaživa do hrobu,
než pověsit kytaru na skobu
a v hospodě znuděně čekat \ldots
\endstrophe

\end{multicols}

\end{document}
