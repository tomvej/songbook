\documentclass{song}
\usepackage[czech]{babel}

\title{Čarodějince z Amesbury}
\author{Asonance}

%% deprecated

\begin{document}

\strophe
       Dm           C           Dm
Zuzana |byla dívka, |která žila |v Amesbury,
          F        C            Dm
s jasnýma |očima a |řečmi pánům |navzdory,
        F         C       Dm            Am
sousedé |o ní říka|li, že |temná kouzla |zná
  B            Am
a |že se lidem |vyhýbá
  B         C       Dm
a |s ďáblem |pletky |má.
\endstrophe

\strophe*
Onoho léta náhle mor dobytek zachvátil
a pověrčivý lid se na pastora obrátil,
že znají tu moc nečistou, jež krávy zabíjí,
a odkud ta moc vychází, to každý dobře ví.
\endstrophe

\strophe*
Tak Zuzanu hned před tribunál předvést nechali,
a když ji vedli městem, všichni kolem volali:
\uv{Už konec je s tvým řáděním, už nám neuškodíš,
teď na své cestě poslední do pekla poletíš!}
\endstrophe

\strophe*
Dosvědčil jeden sedlák, že zná její umění,
ďábelským kouzlem prý se v netopýra promění
a v noci nad krajinou létává pod černou oblohou,
sedlákům krávy zabíjí tou mocí čarovnou.
\endstrophe

\strophe*
Jiný zas na kříž přísahal, že její kouzla zná,
v noci se v černou kočku mění dívka líbezná,
je třeba jednou provždy ukončit ďábelské řádění,
a všichni křičeli jako posedlí: \uv{Na šibenici s ní!}
\endstrophe

\strophe*
Spektrální důkazy pečlivě byly zváženy,
pak z tribunálu povstal starý soudce vážený:
\uv{Je přece v knize psáno: nenecháš čarodějnici žít
a před ďáblovým učením budeš se na pozoru mít!}
\endstrophe

\strophe*
Zuzana stála krásná s hlavou hrdě vztyčenou
a její slova zněla klenbou s tichou ozvěnou:
\uv{Pohrdám vámi, neznáte nic nežli samou lež a klam,
pro tvrdost vašich srdcí jen, jen pro ni umírám!}
\endstrophe

\strophe
~
|Tak vzali Zuzanu na kopec pod šibenici

a všude kolem ní se sběhly davy běsnící,

a ona stála bezbranná, však s hlavou vztyčenou,
                                   G
zemřela tiše samotná pod letní oblo|hou.
\endstrophe

\end{document}
