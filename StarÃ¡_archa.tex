\documentclass{song}
\usepackage[czech]{babel}
\usepackage{multicol}

\title{Stará archa}
\author{Spirituál kvintet}

\begin{document}
\refrain
          C                G\7      C
\[ Já mám |kocábku náram-, |náram-, |náram-,
                G\7   C
kocábku náram-, |náram|nou. \]
\endstrophe

\strophe
C
|Pršelo a blejskalo se sedm neděl, 
                G\7     C
kocábku náram-, |náram|nou, 
C
|Noe nebyl překvapenej, on to věděl, 
                G\7   C
kocábku náram-, |náram|nou. 
\endstrophe

\ref{}

\strophe
C                       F
|Archa má cíl, archa má |směr,
                   G\7   C
plaví se k~Araratu |na se|ver.
\endstrophe

\ref{}

\begin{multicols}{2}
\strophe*
Šem, Nam a Jafet byli bratři rodní, 
kocábku náram-, náramnou, 
Noe je zavolal ještě před povodní, 
kocábku náram-, náramnou. 
\endstrophe

\strophe*
Kázal jim uložiti ptáky, savce, 
kocábku náram-, náramnou, 
'ryby nechte, zachrání se samy hladce,' 
kocábku náram-, náramnou.
\endstrophe
\end{multicols}

\ref{}

\strophe*
Archa má cíl, archa má směr,
plaví se k~Araratu na sever.
\endstrophe

\ref{}

\begin{multicols}{2}
\strophe*
Přišla bouře, zlámala jim pádla, vesla, 
kocábku náram-, náramnou, 
tu přilétla holubice, snítku nesla, 
kocábku náram-, náramnou. 
\endstrophe

\strophe*
Na břehu pak vyložili náklad celý, 
kocábku náram-, náramnou, 
ještě že tu starou dobrou archu měli, 
kocábku náram-, náramnou. 
\endstrophe
\end{multicols}

\ref{}


\end{document}
