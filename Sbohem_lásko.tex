\documentclass{song}
\usepackage[czech]{babel}

\title{Sbohem, lásko}
\author{Waldemar Matuška}

\begin{document}

\strophe
        C      F       G         C\,F\,G
Ať bylo |mně i |jí tak |šestnáct |let,
C               Am          D\7\,G\7
|zeleným údolím |jsem si ji |ved.
     C          C\7           F            Fm
Byla |krásná to |vím a já měl |strach, jak |říct,
           C         G\7             C\,F\,C
když na řas|ách slzu |má velkou jako |hrách.
\endstrophe

\refrain
C\7      F              Dm            Em        Am
|Sbohem, |lásko nech mě |jít, nech mě |jít bude |klid.
       Dm             G\7       C          C\7
Žádnej |pláč už nespra|ví ty mý |nohy toula|vý,
      F              Dm         Em        Am
já tě |vážně měl moc |rád co ti |víc můžu |dát,
       Dm         G\7         C        F      C
nejsem |žádnej ide|ál nech mě |jít zas |o dům |dál.
\endstrophe

\strophe*
A tak šel čas a já se toulám dál,
v kolika údolích jsem takhle stál.
Hledal slůvka, co jsou jak hojivej fáč,
bůh ví co jsem to zač, že přináším všem jenom pláč.
\endstrophe

\ref{}

\begin{recitative}
Já nevím, kde se to v člověku bere -- ten neklid,
co ho tahá z místa na místo, co ho nenechá,
aby byl sám se sebou spokojený jako většina ostatních,
aby se usadil, aby dělal jenom to, co se má,
a říkal jenom to, co se od něj čeká,
já prostě nemůžu zůstat na jednom místě, nemůžu, opravdu, fakt.
\end{recitative}

\refrain*
Lá lá lá \ldots
žádnej pláč už nespraví ty mý nohy toulavý.
\endstrophe

\end{document}
