\documentclass{song}
\usepackage[czech]{babel}

\author{Karel Plíhal}
\title{Kluziště}

\begin{document}

\strophe
C         Em\7   Am\7    C/G      Fmaj\7  C        Fmaj7 G
|Strejček |kovář |chytil |kleště, |uštíp' |z~noční |oblo|hy
C      Em\7   Am\7   C/G     Fmaj\7 C       Fmaj\7 G
|jednu |malou |kapku |deště, |ta mu |spadla |pod no|hy,
C        Em\7 Am\7    C/G     Fmaj\7    C      Fmaj\7 G
|nejdřív |ale |chytil |slinu, |tak šáh' |kamsi |pro pi|vo,
C       Em\7  Am\7 C/G    Fmaj\7 C     Fmaj\7 G
|pak při|táhl |kova|dlinu |a o|brovský |kladi|vo.
\endstrophe

\refrain
      C         Em\7   Am\7        C/G
Zatím |tři bílé |vrány |pěkně za se|bou
      Fmaj\7       C            D\7        G
kolem |jdou, někam |jdou, do ryt|mu se kýva|jí,
      C         Em\7   Am\7        C/G
tyhle |tři bílé |vrány |pěkně za se|bou
      Fmaj\7       C           Fmaj\7     C
kolem |jdou, někam |jdou, nedoj|dou, nedoj|dou.
\endstrophe

\strophe*
Vydal z hrdla mocný pokřik ztichlým letním večerem,
pak tu kapku všude rozstřík' jedním mocným úderem,
celej svět byl náhle v kapce a vysoko nad námi
na obrovské mucholapce visí nebe s hvězdami.
\endstrophe

\ref{}

\strophe*
Zpod víček mi vytrysk' pramen na zmačkané polštáře,
kdosi mě vzal kolem ramen a políbil na tváře,
kdesi v~dálce rozmazaně strejda kovář odchází,
do kalhot si čistí dlaně umazané od sazí.
\endstrophe

\ref{}

\end{document}
