\documentclass{song}
\usepackage[czech]{babel}

\author{Jarek Nohavica}
\title{Lachtani}

\begin{document}

\refrain
   C         F,C  Am        G,C
\[ |Lach lach|,   |lach lach| \]
\endstrophe

\strophe
C               F    C   Am                       G     C
|Jedna lachtaní |rodi|na |rozhodla se že si vyjde |do ki|na
     C                         F      C         Am             G      C
jeli |lodí vlakem metrem a pak |tramva|jí a teď |u~kina Vesmír |lachta|jí
G                C               G                  C         G
|lachtaní úspory |dali dohromady |koupili si lístky |do první |řady
C                          F       C     Am                         G     C
|táta lachtan řekl nebudem |třít bí|du a |pro každého koupil pytlík |araší|dů
\endstrophe

\refrain*
\[ Lach lach, mm, lach lach, mm \]
\endstrophe

\strophe*
Na jižním pólu je nehezky a tak lachtani si vyjeli na grotesky
těšili se jak bude veselo když zazněl gong a v sále se setmělo
co to ale vidí jejich lachtaní zraky sníh a mráz a sněhové mraky
pro veliký úspěch změna programu dnes dáváme film ze života lachtanů
\endstrophe

\refrain*
\[ Lach lach, jéé, lach lach, jéé \]
\endstrophe

\strophe*
Táta lachtan vyskočil ze sedadla nevídaná zlost ho popadla
proto jsem se netrmácel přes celý svět abych tady v kině mrznul jak turecký med
doma zima tady zima všude jen chlad kde má chudák lachtan relaxovat
nedivte se té lachtaní rodině že pak rozšlapala arašídy po kině
\endstrophe

\refrain*
\[ Lach lach, jéé, lach lach, jéé \]
\endstrophe

\strophe*
Tahle lachtaní rodina
od té doby nechodí už do kina
Lach lach
\endstrophe

\end{document}
