\documentclass{song}
\usepackage[czech]{babel}
\usepackage{multicol}

\title{Husita}
\author{Jarek Nohavica}

\begin{document}

\begin{multicols}{2}

\strophe
   C                  G
\[ |Pásával jsem koně |u nás ve dvoře
C              G
|ale už je nepa|su \]
Am             G       C        F
|Chudák ten je |dole a |pán naho|ře
C            G    Am
|všeho jenom |doča|su
F   C            G    C
|jó |všeho jenom |doča|su
\endstrophe

\strophe*
\[ Máma ušila mi režnou kytlici
padla mi jak ulitá \]
Táta vytáh ze stodoly sudlici
a teď seš chlapče husita
jó a teď seš chlapče husita
\endstrophe

\refrain
   C              G       C
|Hejtman volá: |Do zbra|ně!
Am     Dm    G7      C_G7
|Bijte |pány |hrr na |ně!
   C              G       C
|Hejtman volá: |Do zbra|ně!
Am     Dm    G7      C
|Bijte |pány |hrr na |ně!
Am     Dm     G7
|A mně |srdce |buší
C               Am
|lásce dal jsem |duši
Dm             G     C
|jen ať s námi |zůsta|ne
\endstrophe

\strophe*
\[ U města Tachova stojí křižáci
leskne se jim brnění \]
Sudlice je těžká já se potácím
dvakrát dobře mi není
jó dvakrát dobře mi není
\endstrophe

\strophe*
\[ Tolik hezkejch holek chodí po světě
já žádnou neměl pro sebe \]
Tak si říkám: Chlapče křižák bodne tě
a čistej půjdeš do nebe
jó čistej půjdeš do nebe
\endstrophe

\ref{}

\strophe*
\[ Na vozové hradbě stojí Marie
mává na mě zdaleka \]
Křižáci kdo na ni sáhne mordyjé
ten se pomsty dočeká
jó ten se pomsty dočeká
\endstrophe

\strophe*
\[ Chtěl jsem jí dát pusu tam co je ten keř
řekla: To se nedělá! \]
Když mě nezabijou to mi holka věř
budeš moje docela
jó budeš moje docela
\endstrophe

\ref{}

\strophe*
\[ Už se na nás valí křižáci smělí
zlaté kříže na krku \]
Jen co uslyšeli jak jsme zapěli
zpět se ženou v úprku
jó zpět se ženou v úprku
\endstrophe

\strophe*
\[ V trávě leží klobouk -- čípak mohl být?
Prý kardinála z Anglie \]
Tam v té trávě zítra budeme se mít
já a moje Marie
jé ať miluje kdo žije
jé ať žije historie
\endstrophe

\end{multicols}

\end{document}
